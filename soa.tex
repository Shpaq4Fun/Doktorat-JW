\chapter{State of~the~art}

Local damage detection at~the~early stage of~development is~one of~the~most important tasks in~modern condition monitoring. Management of~the~machines is~heavily influenced by~this aspect, which utilizes broad spectrum of~methods, often interdisciplinary ones. Since author focuses on~fault detection based on~vibration and temperature data, this chapter is~divided into two parts.

\section{Vibration-based damage detection}
Multiple works focused on~the~local damage detection in~the~bearings and gearboxes can be~found in~the~literature. It was proven by~Cempel, Jonak and many other authors that multidimensional analysis for condition monitoring, especially under nonstationary operations is~very interesting and fruitful approach \cite{cempel2007multidimensional,zimroz2013two,bartkowiak2014dimensionality,zimroz2014diagnostics,jedlinski2015early,jedlinski2017disassembly}. 

% bartelmus
In the~field of~industrial machine diagnostics, one can find multiple different types of~devices. One of~the~most difficult types to~analyze are those working in~highly nonstationary conditions. Bearings of~such machines are exposed to~high amount of~stress and often tend to~wear very quickly. Difficulty in~analysis of~such signals comes from the~variation of~vibration-based diagnostic features caused mostly by~load/speed variation and low signal-to-noise levels. Zimroz, Bartelmus, Barszcz and Urbanek in~\cite{zim_barszcz_mssp} proposed method based on~the~statistical patter recognition for bad and good condition of~the~bearings. Authors have extended this method by~load susceptibility characteristics as~a~feature. Interesting work has been also presented by~Burdzik where authors discuss multidimensional techniques for the~analysis of~nonstationary vibration signals from the~automotive and transportation industry \cite{burdzik2019multidimensional,burdzik2018application,burdzik2013research}.

Paper that is~still one of~the~most important ideas in~the~field is~\cite{bartelmus}. In this article, authors introduced new diagnostic feature which could be~used in~time-varying operating conditions. This method exploits the~fact that a~planetary gearbox in~bad condition is~more susceptible to~load than a~gearbox in~good condition. One needs to~capture signals for different external load values and calculate a~simple spectrum based feature versus operating conditions indicator (current or instantaneous rotation speed). In a~certain range of~operating conditions the~diagnostic relation (i.e. the~dependence between the~spectral features and the~operating conditions indicator) is~linear. However, since a~gearbox in~bad condition is~more susceptible to~load than the~gearbox in~good condition the~relation will be~different for the~two cases. Using a~simple regression equation one can calculate the~slope of~the~straight line, which expresses the~new diagnostic feature.

One of~the~most prolific reviews in~the~field was put forth by~Samuel and Pines \cite{samuel_pines}. It recapitulates over two decades of~development of~the~diagnostic algorithms for the~gearboxes. It describes widely used statistical measures for the~signal energy. Moreover, authors compare application of~various time-frequency representations such as~spectrogram or Wigner-Ville distribution for the~problem of~local damage detection \cite{forrester, forrester2, samuel2, nasa1}. 

% podstawy bearings
McFadden and Smith \cite{mcfadden_bearings} presented one of~the~most significant research pieces regarding the~local damage detection of~the~bearings. The investigated the~high-frequency resonance approach, as~suitable for the~vibration-based monitoring of~gearboxes because of~the~possibility to~separate the~signature of~a~defective bearing from the~rest of~the~signal. Another interesting techniques regarding fault detection in~bearings were published in~\cite{AYE20151779,COCCONCELLI2012667}.

% Time averaging 
Time-synchronous averaging is~another successful method that is~commonly used in~the~field. Vibration signal is~averaged within periodic sections, which unfortunately requires prior knowledge of~the~correct period, that can be~a~disadvantageous property regarding the~robustness of~the~procedure \cite{braun_tsa,Wang201774}.  In the literature one can also find application of the transient signal analysis and unsupervised classification \cite{TIMUSK20081724} for local damage detection in industrial machinery. 

%STFT
Very often time-frequency representations serve as~a~basis for multidimensional diagnostic procedures based vibration data analysis, with Short-Time Fourier Transform being arguably the~most popular one \cite{oppenheim1999discrete}. a~modification of~the~STFT representations was proposed by~Obuchowski in~\cite{obuch3}. The problem is~that in~most (if not all) of~the~industrial cases signal-to-noise ratio is~very low. Damage-related modulations are buried deep in~high energy of~the~background noise. In order to~deal with this issue, authors have proposed the~local maxima approach. It is~based on~looking for the~local maxima in~the~spectrogram. Proper derivation of~the~weight vector for the~spectrogram can significantly improve visibility of~the~impulsive components. 

% falki
Wavelet analysis is~another topic strongly expanded in~the~recent years \cite{WANG1996927, staszewski, samuel3}. An~interesting approach was presented in~\cite{linzuo}, where an~adaptive filter based on~the~Morlet wavelet has been developed. Adaptiveness of~the~wavelet is~achieved by~allowing the~parameters to~not be~fixed. In \cite{rozsz_jve_16} Peng and Chu review the~wavelet-related techniques used in~the~field of~condition monitoring. In \cite{WANG1996927} Wang and McFadden proposed the~application of~this method for fault detection in~helicopter gearboxes. Miao and Makis \cite{MIAO2007840} have investigated an~interesting approach that utilizes both probabilistic and wavelet techniques. They have classified the~machinery state, for such knowledge allows one to~schedule appropriate maintenance in~practice based on~the~hidden Markov models and validation using wavelet modulus maxima distribution. Another research concerning application of~wavelets and time-frequency decomposition were presented in~\cite{ALBADOUR20112083, RUBINI2001287}. In \cite{combet2012novel} Gelman and Combet propose a~technique called instantaneous wavelet bicoherence in~application to~vibration measurements for local damage detection in~gears. It allowed to~detect gear pitting in~both experimental and real-life scenarios. Presented technique demonstrated superior capablities over the~conventional detection methods based on~the~wavelet transform. In \cite{peter2001wavelet} Tse presents the~application of~wavelet analysis to~the~problem of~rolling element bearing diagnostics, and in~\cite{PENG2005974} compares wavelet transform to~Hilbert-Huang transform in~application to~the~same problem.

% Cyclo
Other class of~tools for information discovery are so-called cyclostationary maps: cyclic spectral coherence (CSC), cyclic spectral density (CSD) and cyclic modulation spectrum (CMS) \cite{antoni3}. Cyclic modulation spectrum is~bi-frequency representation defined as~a~combination of~Fourier transforms of~instantaneous power spectra of~the~signal calculated using discrete Fourier transforms of~the~absolute value of~the~STFT. Another bi-frequency map is~a~cyclic spectral density, defined as~a~correlation density of~two spectral components spaced apart by~frequency $\alpha$ in~the~vicinity of~the~central (carrier) frequency \textit{f}. If the~spacing is~equal to~0, CSD reduces to~the~ordinary power spectrum. Finally, cyclic spectral coherence is~a~normalized CSD. All of~described representations play an~important role in~cyclostationarity analysis in~the~signals. However, their main disadvantage is~the~fact that they use the~second moment which can be~undefined for the~impulsive behavior of~the~data. 

% SCD
One of~the~most recent approaches to~damage detection was put forth by~Urbanek \cite{urbanek2014integrated}, \cite{URBANEK2012399} where authors studied generalization of~the~spectral correlation density, known as~modulation intensity distribution. Authors have investigated its functionality as~a~descriptor for detection of~the~presence of~rolling element bearing faults in~noisy environments. 

% SK
Significant amount of~research conducted in~recent years was aimed at~the~fault detection in~the~presence of~Gaussian noise. Antoni and Randall have put forth very interesting approach based on~very simple statistic \cite{antoni_randall}. This presents a~detection method using spectral extension of~the~sample kurtosis. In such a~way it~can represent a~filter magnitude characteristics or a~tool for informative frequency band selection within which further analysis should be~conducted. However, it~assumes the~additive Gaussian noise in~the~signal, so its application is~limited to~theoretical considerations. In addition kurtosis is~very sensitive to~rare or even singular impulses known as~artifacts, which additionally jeopardizes the~correctness of~the~results. 

Spectral kurtosis could also serve as~a~tool to~construct optimal Wiener filter as~shown by~Gelman and Combet \cite{combet2009optimal}. Authors applied it~to~gear residual signal after removing meshing component, which allowed to~detect surface pitting on~early stage of~development.

% Kurtogram
Modification of~the~spectral kurtosis method is~a~kurtogram \cite{antoni2007fast}. It is~based on~the~biforcation of~frequency domain into gradually smaller subbands. Input data is~then processed with a~bandpass filter corresponding to~a~particular subband and kurtosis of~the~result is~calculated. Composition of~those kurtosis values into the~array of~kurtogram allows user to~define the~most impulsive subband. However, the~disadvantages of~the~kurtosis persist. 

%Protrugram
Kurtogram was also a~basis for the~design of~the~techniques that will not suffer from singular impulses. Study put forth by~Barszcz and Jabłoński in~\cite{barszcz_mssp} with singular impulse influecing the~kurtogram-related analysis proved it~to~be very ineffective in~the~presence of~random disturbances. Hence, the~enhanced version of~the~kurtogram was developed by~the~name of~protrugram. It relies on~the~kurtosis of~the~envelope spectra instead the~kurtosis of~the~filtered time series.

%zaku
Antoni and Borghesani in \cite{BORGHESANI2017378} investigated the influence of the $\alpha$-stable noise on the traditional statistical estimators, and proposed log-envelope analysis as a countermeasure to nonexistent second moments. An interesting approach of vibration-based damage detection inspired by those studies was presented by Żak. In \cite{11419466920160301} he proposed to use the parameters of $\alpha$-stable distribution as a basis to construct a custom selector either to serve as a filter for the input data, or as an indicator of the informative frequency band. He also proposed to use a multidimensional data representation based on time-frequency representations, called Fractional Lower-Order Covariance (FLOC), for the purpose of machine diagnostics \cite{Zak2014}. It exhibits advantageous properties in comparison to classic time-frequency representations, because of the focus on enhancing periodic features.

Besides damage detection specifically, wear monitoring and detection in~rotating machinery is~another very important aspect of~diagnostics. Heyns et.al. have presented multiple examples of~analytical techniques based not only on~vibration data analysis, but also fusing it~with other features, e.g. strain measurement, angular speed etc. \cite{stander2002using,scheffer2001wear} using techniques such as~order tracking \cite{wang2011combined,wang2012application}, neural networks \cite{herzog2009machine,ngwangwa2014reconstruction}, or Vold-Kalman filtering \cite{wang2009vold,wang2011combined}.

\section{Temperature-based damage detection}

In recent years one can observe increasing interest towards automated temperature-based machine monitoring and diagnostics. Especially in~the~context of~the~philosophy of~the~"Industry 4.0", SCADA systems are becoming more and more common in~industrial facilities, which enables the~implementation of~such solutions \cite{zhang2012research,eliasson2013internet,bongers2008fault,wilkinson2014comparison}.

% coolant temperature modeling
In \cite{junnuri2015engine} Junnuri describes developed an~internal combustion engine coolant temperature model for control and diagnostics development and validation purpose. Authors propose to~use both physical laws and dynamic data to~develop the~dynamic model of~the~cooling system.

In \cite{sawicki2015automatic} Sawicki proposed a~multichannel technique for temperature analysis from belt conveyor gearboxes. Authors utilize Wold's decomposition for model-based anomaly detection. Temperature modeling is~also used by~Guo, Infield and Yang in~\cite{guo2011wind} 
where they propose a~condition-monitoring method based on~the~nonlinear state estimate technique for a~wind turbine generator. Authors construct the~normal behavior model of~the~electrical
generator temperature for the~purpose of~trend analysis for deviation detection. Trend-based anomaly detection was also described in~\cite{astolfi2014fault} with respect to~wind turbines.

In \cite{Nembhard2013} the~method for bearing fault diagnosis based on~the~conjunction of~temperature and vibration data is~presented. Authors indicate that the~usage of~information from both types of~data yields better results than the~method based only on~vibration data analysis \cite{Nembhard2013b}.

An interesting approach to the temperature analysis was presented by Staszewski and Dao in \cite{DAO2018107}. Authors propose to use the approach of SCADA data cointegration applied to wind turbines diagnostic signals. This approach allowed to analyze nonlinear data trends to reliably detect abnormal problems.

Finally, thermographic image analysis is also very potent basis for machine monitoring and diagnostics \cite{Lim2014,5608306}. It is a very powerful approach and as an image-analysis-related method it can be also understood as a multidimensional one, however this approach is not the subject of this dissertation.

