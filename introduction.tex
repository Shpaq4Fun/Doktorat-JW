\chapter{Introduction}

In this dissertation, the author focuses on~the~problem of~failure detection in~mining machines. Time-varying and harsh environmental conditions determine high workload, so effectiveness and availability demands are big challenges for maintenance staff. Due to~the~type of~work performed by~the~mining machines and difficult operating conditions themselves, strategic components of~the~machines are often exposed to~the~higher risk of~the~failure. One of~the~most important issues of~self-propelled machines used in~underground mines is~related to~engine overheating detection. The problem has been indicated by~maintenance staff in~the~mining company. Unfortunately simple decision making based on~thresholding-related methods is~absolutely ineffective due to~time-varying load and environmental impact.

On the~other hand, stationary machines (e.g. electric engines, gearboxes, etc.) suffer heavily from local and distributed damage occurrences. Both of~those types of~problems are the~main cause of~unjustifiable stoppages and unwanted disturbances in~production.

Damage detection is~a~very interesting topic for insurance companies \cite{gellermann2003requirements,lloyd2007richtlinien}. Knowledge about the~expected frequency of~failures and the~ability to~predict them heavily influences insurance rates for the~machine. On the~other hand, the~progress of~reliable diagnostic methods can allow planning the~maintenance schedules better, which is~especially important when the~mine possesses a~lot of~machines and it~is~impossible to~service even small fraction of~them at~once. It can allow reducing unnecessary downtime situations to~a~minimum.

Furthermore, the~development of~machine diagnostics can be~beneficial from the~safety point of~view. Sudden and unforeseen failure during operation can cause more damage to~the~machine, which in~turn can endanger operator, as~well as~other people and machines.

In this dissertation, the~author presents novel analytical methods designed for diagnostics of~heavy-duty mining machinery, both mobile (self-propelled machines) and stationary (drive units, crushers), as~well as~specific components (e.g. gears and bearings).