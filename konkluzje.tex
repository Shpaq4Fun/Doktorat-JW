\chapter{Conclusions}


Research conducted by the author is focused on the application of multidimensional data processing techniques to the problem of technical diagnostics. The data used in the thesis were mostly temperature and vibration (acceleration) signals, however the most significant emphasis was put on vibration-based local damage detection in gearboxes and bearings. The most important aspect of this work was to develop novel set of tools suitable for the task of~change of~condition detection in heavy-duty machinery used in mining industry. 

Developed algorithms for data analysis covers such steps as: preprocessing, validation, data transformation, design diagnostic feature (single or multiple ones), detection of~the change of~technical condition, indicating that change in original signal (usually invisible in raw data) and validation/comparison with existing techniques or confirmation via visual inspection or~discussion with maintenance crew.

In most of the cases change in condition was hidden in the signal (damage-related low energy impulses in vibration signals) or was hard to detect in daily data or long-term observation (temperatures). As mentioned, due to high level of noise, the presence of non-Gaussian noise (crusher), complexity of the machine, influence of operational regimes (cooling down of machines in every shift and weekends), weak signature of change of technical condition, local problems in data acquisition, transmission, and many other factors, it has been proven that condition monitoring using more channels and multidimensional data representation of those, significantly improves efficiency of the diagnostic procedures.

In case of temperature, the improvement could be evaluated in a subjective way (one can see that there is clear change in proposed diagnostic feature). On the other hand, in case of vibration signals, a statistical measure of impulsiveness was used (i.e. kurtosis) to prove efficiency of~proposed techniques.

\section{Temperature data analysis}

In section \ref{temp_ica} author presents a~method for technical condition change detection using long-term multichannel temperature data from heavy-duty gearboxes operating together in~a~single driving station of~a~belt conveyor. The change was manifested by~the~sudden increase of~the~temperature on~one of~the~gearboxes, however it~was very difficult to~notice that in~the~long-term record. Utilization of~independent component analysis allowed to~construct a~diagnostic feature that clearly revealed the~moment of~state change allowing to~properly locate it~in~time (see section \ref{result_ica}). \textbf{The obtained result has been confirmed by the information obtained from the CMMS system in the mine. The character of the condition change determined that it was impossible to be detected based only on the observation of raw signals.}

In section \ref{temp_em} author presents a~method for operational regime distinction of~the~gearbox of~the~belt conveyor based on~single-channel long-term temperature record. Technique is~based on~multidimensional parameterization of~the~input data which leads to~classification of~behavior within individual days using expectation-maximization approach. This way it~was possible to~automatically assess when the~machine was operating in~the~incorrect technical condition (see section \ref{result_em}). 

In section \ref{temp_bulgaria} author presents a~method for technical condition change detection in~the~self-propelled mining machine, in~this case load-haul-dump machine. The analysis is~performed based on~single-channel long-term record of~the~engine coolant temperature. Transforming time-domain signal to~multidimensional time-time representation allowed to~analyze data distributions shift by~shift, which leads to~identification of~segments that can be~characterized with different statistical properties. Analysis of~those properties allows to~perform segmentation to~differentiate dissimilar regimes and analyze the~process of~occurring change of~technical state of~the~machine (see section \ref{real_bulgaria}). Detected state change was connected to~machine overheating caused by~the~cooling unit being overly plastered by~mud. \textbf{It was not possible to detect those changes directly based only on the temporal observation of the temperature signal due to high variability if the load and strong ambient temperature impact. The results have been validated by the CMMS system of the mine.}

The last method related to~temperature data analysis is~describe in~the~section \ref{temp_ad}. It is~also connected to~technical condition change detection in~the~self-propelled mining machine, but assumes different approach to~the~analytical process. Similarly to~the~previous technique, input data is~transformed to~time-time representation, but in~this approach the~core of~analysis is~based on~statistical testing of~the~temperature distribution functions. Anderson-Darling statistic is~used as~an~elementary tool to~construct two-dimensional feature map describing relations between machine behavior during particular work shifts. Segmentation of~this map enabled identification of~the~regime switch moments in~time domain, which allowed to~locate when the~failure became critical and when the~machine was serviced (see section \ref{real_ad}). In addition, comparison of~the~results with the~ones provided by~previously described method allowed to~conclude that obtained answers were exactly the~same with respect to~the~absolute time. Dissimilarities between the~relative time points indicated by~both methods are a~result of~different preprocessing approach which produced different relative time bases. \textbf{Similarly to the previous case the results were impossible to obtain without the application of dedicated analytical method, and results were validated with the information in CMMS system.} 

In case of the temperature-based analysis, it was not possible to compare the results with the ones obtained by previously used analytical methods, because none of the results has been obtained before for the regarded machines. No practical evaluation of the registered data performed by the maintenance crew allowed to properly assess the described issues.

\section{Vibration data analysis}

In section \ref{met_pca} author presents a~method for local damage detection in~heavy-duty gearbox. Described technique performs multidimensional analysis of~time-frequency representations of~multichannel vibration signal. The usage of~principal component analysis in~the~analysis of~such multidimensional data structure allowed to~identify local damage on~the~gear wheel of~the~second shaft of~the~gearbox (see section \ref{result_pca}). Additionally author compared the~results with the~analogous methodology applied to~individual channels of~he data set and evaluated the~result using the~kurtosis value of~the~resulting output signals (see Tab. \ref{tab:wnoski_tab1}). \textbf{As a~result, it~has been proven that multichannel data analysis provides complementary information that is~distributed among the~channels, which in~turn allows to~obtain better quality of~the~result compared to~single channel analysis.}

\begin{table}[ht!]
  \centering
  \caption{Kurtosis values for single-channel vs multichannel results.}
  \begin{tabular}{|l|l|}
  \hline
     \textbf{Channel} & \textbf{Kurtosis} \\ \hline
     1 & 4.06 \\ \hline
     2 & 6.15 \\ \hline
     3 & 10.34 \\ \hline
     4 & 7.82 \\ \hline
     Mulichannel & 22.83 \\ 
  \hline
  \end{tabular}
  \label{tab:wnoski_tab1}
\end{table}


In section \ref{met_nmf_enc} author describes a~method for local damage detection of~a~rolling-element bearing operating in~a~drive pulley of~the~belt conveyor. Presented technique takes advantage of~nonnegative factorization of~a~spectrogram matrix constructed for a~single-channel vibration signal. Usage of~the~feature matrix obtained using NMF algorithm called \emph{encoding matrix} allowed to~isolate the~pattern describing cyclic impulsive component related to~the~failure identified as~local damage on~the~outer race of~the~bearing (see section \ref{result_enc}). Author compared obtained result with the~signal processed with spectral kurtosis, which is~a~classic technique for impulsive component extraction in~vibration signal analysis. \textbf{As a~result, kurtosis value of~output signal obtained by~the~described method (260) was significantly higher than for the~reference method (15).} Such result indicates much better quality and clarity of~the~extracted diagnostic component in~favor of~the~proposed method.

In section \ref{met_nmf_base} author describes a~method for local damage detection of~a~rolling-element bearing operating in~a~copper ore crusher. The analysis was again performed on~the~time-frequency representation of~the~single-channel vibration data. However, in~this case the~problem regarded the~extraction of~fault-related low-energy cyclic component from the~signal dominated by~high-energy non-periodic impulsive noise. In this case the~other feature matrix produced by~NMF algorithm was used, called \emph{base matrix}. It allowed to~identify the~informative frequency band occupied by~the~component of~interest, but also to~create the~filter that enabled the~extraction of~regarded component from the~input signal. \textbf{As a~result, the~relevant component has been isolated in~a~clear way, which allowed to~confirm the~damage in~the~bearing (see section \ref{result_base}). The attempt to~compare the~result with processing with spectral kurtosis was unsuccessful, since the~SK method failed because of~the~dominating sensitivity to~incorrect high-energy impulsive noise. This in~turn confirmed the~superiority of~the~presented technique.}

In section \ref{met_nmf_both} author describes a~method for local damage detection of~a~rolling-element bearing on~the~shaft of~a~reciprocating gas compressor operating on~the~offshore oil rig. Since the~cyclic components present in~the~signals are characterized with higher frequencies relative to~the~previously described cases, time-frequency representation has been disregarded as~a~suitable basis for the~analysis. In this case a~bi-frequency representation called \emph{cyclic spectral coherence} was used. Additionally the~way that the~machine operates involves expected cyclic components originating from the~operation of~pistons, and component of~interest related to~the~bearing damage is~also cyclic. Because of~that this technique consists of~several stages of~processing for precise information extraction. It involves utilization of~both base and encoding matrices produced by~factorizing the~bi-frequency matrix, as~well as~spatial denoising and Monte Carlo iterations for feature quality enhancement (see section \ref{result_both}). \textbf{Such multistage processing allowed to~identify the~fault as~local damage on~the~inner race of~the~bearing.} Additionally the~obtained diagnostic feature that is~equivalent in~the~interpretation to~the~envelope spectrum, reveals better quality and clarity in~comparison to~the~results obtained in~previous research paper that analyzes this data \cite{barszcz2013bearings}. \textbf{Comparison with the classical method was impossible, because the techniques like Spectral Kurtosis failed completely for the assessment of the fault in the regarded data. However, the subjective comparison of the envelope spectra with the ones obtained in the previous works regarding the analysis of this data set indicates the high precision and overall superiority of the proposed method.}

Finally the~section \ref{met_pga} presents the~novel approach to~optimal filter design using the~original technique called \emph{progressive genetic algorithm}. In this case author investigated the~previously mentioned vibration signal measured on~a~rolling-element bearing operating in~a~drive pulley of~the~belt conveyor. In this approach the~aim was to~design a~digital filter in~such a~way that it~would be~optimal for the~given input data. While ordinary genetic algorithm or other optimization method can control the~shape of~designed filter, progressive modification proposed by~the~author is~also capable of~determining the~optimal filter length, and hence its level of~detail and precision. Additionally, optimization-based filter design methods proposed by~other authors require providing additional information about the~desired filter, while approach proposed by~the~author does not require any prior knowledge about the~input data or the~desired outcome. \textbf{Obtained optimal filter allowed to~extract the~impulsive component related to~the~failure, which is~the~local damage on~the~outer race of~the~bearing.} Additionally result has been compared with classical methods: spectral kurtosis filtration and kurtogram filtration. Evaluation was based on~the~kurtosis value of~upper envelope of~the~obtained filtered components. \textbf{The kurtosis value of~129 for PGA method shows significantly better quality of~result in~comparison with kurtogram (kurtosis=51) and spectral kurtosis (kurtosis=38).}


